\section{Introduction}

The formation of the RoboNotts team stems from a goal to afford students practical exposure including the development of robust and dependable assistive robotics. These technologies are designed to address the challenges associated with supporting individuals in diverse scenarios, such as older adults living independently and patients in healthcare settings. The RoboNotts team operates within the Cobot Maker Space (CMS), an expressly designed area for collaborative robotics research at the University of Nottingham in Nottingham, United Kingdom. The RoboNotts team collaborates with researchers and students, serving as a dedicated testing ground for innovative research initiatives.\\
The RoboNotts team is comprised of members from a range of backgrounds, united with a common interest in robotics, and belief that assistive robotics will be an important aspect of supporting vulnerable populations in the future. The team is predominantly composed of undergraduate students from the University of Nottingham, mainly in the school of computer science, electrical engineering, and mechanical engineering. The team also includes PhD students, technicians and research associates working at the CMS. The RoboNotts team's main aim is to provide visibility to this realm of robotics, and a platform for undergraduate students to learn the skills required to thrive in robotics. \\
Structured into several specialised sub-teams, each dedicated to specific aspects of robotic development, the RoboNotts team operates under the leadership of a designated team leader. This leader assumes comprehensive responsibility for integration and oversees all organisational tasks, with support from the CMS. The team benefits from access to the 'Living Space', a designated area simulating a small flat. This space serves as an environment for the development, testing, and implementation of assistive robotics and other sensor-based systems within a home setting. The CMS has previously hosted assistive robotics competitions. 